\documentclass[tikz,svgnames]{standalone}
\usepackage[fontset=fandol]{ctex} % 若需要,用 XeLaTeX 编译添加中文支持
\usetikzlibrary{mindmap}

\begin{document}
\begin{tikzpicture}[
    mindmap, every node/.style=concept, concept color=orange, text=white,
    level 1/.append style={level distance=5cm, sibling angle=60, font=\LARGE},
    level 2/.append style={level distance=3.5cm, sibling angle=45, font=\large}
  ]

  \node{\huge{\textsf{数据分析}}} [clockwise from=60]
  child [concept color=DarkMagenta] {
      node {\textit{数据准备}} [clockwise from=120]
      child { node {数据对象}}
      child { node {数据获取}}
      child { node {数据清洗}}
      child { node {数据操作}}
    }
  child [concept color=DarkBlue] {
      node {\textit{数据探索}} [clockwise from=30]
      child { node {ggplot2 入门}}
      child { node {基础图形}}
      child { node {统计图形}}
    }
  child [concept color=Brown] {
      node {\textit{数据交流}} [clockwise from=-30]
      child { node {交互图形}}
      child { node {交互表格}}
      child { node {交互应用}}
    }
  child [concept color=teal] {
      node {\textit{统计分析}} [clockwise from=-75]
      child { node {统计检验}}
      child { node {回归分析}}
      child { node {功效分析}}
    }
  child [concept color=purple] {
      node {\textit{数据建模}} [clockwise from=-120]
      child { node {网络分析}}
      child { node {文本分析}}
      child { node {时序分析}}
    }
  child [concept color=DarkGreen] {
      node {\textit{优化建模}} [clockwise from=180]
      child { node {统计计算}}
      child { node {数值优化}}
      child { node {优化问题}}
    };
\end{tikzpicture}
\end{document}
